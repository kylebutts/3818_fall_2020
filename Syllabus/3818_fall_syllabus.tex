\documentclass[12pt]{article}

\input{syllabus_preamble.tex}

\usepackage{hyperref}


\date{\today}

\begin{document}

\section*{ECON 3818: Intro to Statistics with Computer Applications}
\begin{center}
	MWF 11:30-12:20 \\ Room: Holmes 141
\end{center}

\vspace{.2cm}
\noindent \textbf{Instructor:} Kyle Butts

\noindent \textbf{E-mail:} \href{mailto:kyle.butts@colorado.edu}{kyle.butts@colorado.edu}

\noindent \textbf{Office Hours:} M 9:00-11:00 \& T 10:00-12:00 and by appointment

\vspace{5mm} \noindent \textit{Please allow me 24 hours to respond to all emails. If you need help with course material, please see me during my office hours. If you need more help than can be provided in office hours, consider visiting the department's undergraduate tutor or viewing the department's private tutor list:} \url{http://www.colorado.edu/econ/undergraduate/tutor_list.pdf}

\section*{Course Description}
Econ 3818 is an introductory course in the theory and methods of statistics. Statistics allows datasets to be transformed into usable information, analyzed for patterns and trends, which improve decision-making.

Upon completion of the course, students should be able to

\begin{itemize}
    \item Be prepared for a future course in Econometrics -- the data driven side of economics.
    \item Will be able to load datasets into R and perform statistical methods to gather information about the data.
    \item Understand the probability theory behind basic statistical tests and implement the methods.
\end{itemize}
    
We will study basic probability, probability distributions (especially the normal distribution), and descriptive and inferential statistics, including estimation and hypothesis testing. Emphasis is on both theory and applications. Weekly problem sets will explore issues in statistical theory and practice. The course will use the programming language R to do data analysis on simulated and real datasets

\vspace{5mm}\noindent 
\textbf{Required Materials}
\begin{itemize}
	\item Textbook: \textit{The Basic Practice of Statistics}. David Moore, William Notz, and Michael A Fligner.  
\end{itemize}

\noindent 
\textbf{Prerequisites:}
\begin{itemize}
	\item Econ 2010 \& 2020. Econ 1088 (or an approved similar course). This class requires algebra and calculus so exposure to these concepts is required. 
\end{itemize} 


\noindent 
\textbf{Computer Application:}
\textbf{R} is a free programming language that is utilized primarily for data analysis. We will spend time throughout the course working on \textbf{R} exercises through the \textbf{RStudio} interface. 

\vspace{5mm}\noindent 
\textbf{Grading:}
\begin{center}
	\small{
        \begin{tabular}{| l | c |}
            \hline
            \textbf{Assignment} & \textbf{Percentage}\\ 
			\hline
			Homework & 10\% \\ \hline
			R Problem Sets & 10\% \\ \hline 
			R Project & 10\% \\ \hline 
			Midterm 1 & 20\% \\ \hline
			Midterm 2 & 20\%  \\ \hline
			Final & 20\%  \\ \hline
		\end{tabular}}
\end{center}


\noindent 
\textit{Curving:} Midterms may be curved individually, and a curve may be applied to the overall course grade to conform to departmental standards. I will automatically increase final course grades that are 0.5\% below any grade cutoff after any final grading curve has been applied.

\noindent
\textit{Grade Adjustments:} Other than the 0.5\% bump discussed above, I will not grant any request to increase your grade to meet a certain cutoff. You will receive the grade that you earned throughout the course. If you are concerned about your grade(s) you should immediately come talk to me. I will do everything I can to help you be successful in this course.
 
\vspace{5mm}\noindent 
\textbf{Grading Scale:}
\begin{center}
	\footnotesize{
		\begin{tabular}{|c|c|c|c|}
			\hline
			\textbf{Grade} & \textbf{Percentage} & \textbf{Grade} & \textbf{Percentage}       \\ \hline
			A  & $93 \leq x$      & C  & $73 \leq x < 77$ \\ \hline
			A- & $90 \leq x < 93$ & C- & $70 \leq x < 73$ \\ \hline
			B+ & $87 \leq x < 90$ & D+ & $67 \leq x < 70$ \\ \hline
			B  & $83 \leq x < 87$ & D  & $63 \leq x < 67$ \\ \hline
			B- & $80 \leq x < 83$ & D- & $60 \leq x < 63$ \\ \hline
			C+ & $77 \leq x < 80$ & F  & x $< 60$         \\ \hline
		\end{tabular}}
\end{center}


\vspace{5mm}\noindent 
\textbf{Homework:}
There will be weekly homework assignments assigned through the Sapling website. These will be due by 11:59 pm on \textbf{most} Sundays. % You can access assigned homework problems through \url{http://www.saplinglearning.com/ibiscms/} and using code ``butts". 
No late homework will be accepted. The 4 chapter homeworks with the lowest grades will be dropped. 
 
\vspace{5mm}\noindent 
\textbf{Recitation:} Recitation attendance is not mandatory. However, it is crucial for success to attend recitation. Material covered in recitation will look very similar to exam questions and will serve as high-quality review of lecture material. Your TA is PhD student Xiang Chi. Email: \href{mailto:xiang.chi@colorado.edu}{xiang.chi@colorado.edu}.
 
\vspace{5mm}\noindent 
\textbf{R Project \& Exercises:} There will be five simple \textbf{R} assignments and one data project throughout the semester. The R project will be worth 10\% and the R assignments will be worth another 10\%. R assignments must be turned in to Canvas before class starts on Wednesday. A `.Rmd' (R Markdown file) and/or a `.pdf' should be turned in for the assignment. 
 
\vspace{5mm}\noindent
\textbf{Exams:}
There will be two midterms throughout the semester. They will consist of multiple choice questions along with a couple of free response questions. You may use your notes and book for the exam, but may not work with anyone on them. Any tables required will be provided by the instructor. There will be no make-up exams, unless there is documentation of a medical or family emergency. If you miss an exam, the weight of that exam will be added to the final exam.The final exam is cumulative, but the midterms are not.

%\vspace{5mm}\noindent
%\textbf{Extra Credit:}
%The only extra credit opportunity will be through iClickers. There will be roughly 2-3 clicker questions per lecture. A maximum of five percentage points will be added to your grade for excellent clicker participation. Make sure to register your iClickers, instructions provided here: \\ \url{https://oit.colorado.edu/tutorial/cuclickers-iclicker-remote-registration}

\section*{COVID-19 Information:}

As a matter of public health and safety due to the pandemic, all members of the CU Boulder community and all visitors to campus must follow university, department and building requirements, and public health orders in place to reduce the risk of spreading infectious disease. Required safety measures at CU Boulder relevant to the classroom setting include:

\begin{itemize}
	\item Maintain 6-foot distancing when possible,
	
	\item Wear a face covering in public indoor spaces and outdoors while on campus consistent with state and county health orders,
	
	\item Clean local work area,
	
	\item Practice hand hygiene, 
	
	\item Follow public health orders, and
	
	\item If sick and you live off campus, do not come onto campus (unless instructed by a CU Healthcare professional), or if you live on-campus, please alert CU Boulder Medical Services.
\end{itemize}
	
Students who fail to adhere to these requirements will be asked to leave class, and students who do not leave class when asked or who refuse to comply with these requirements will be referred to \href{https://www.colorado.edu/sccr/}{Student Conduct and Conflict Resolution}. For more information, see the policies on \href{https://www.colorado.edu/policies/covid-19-health-and-safety-policy}{COVID-19 Health and Safety} and \href{http://www.colorado.edu/policies/student-classroom-and-course-related-behavior}{classroom behavior} and the \href{http://www.colorado.edu/osccr/}{Student Code of Conduct}. If you require accommodation because a disability prevents you from fulfilling these safety measures, please see the “Accommodation for Disabilities” statement on this syllabus.

Before returning to campus, all students must complete the \href{https://www.colorado.edu/protect-our-herd/how#anchor1}{COVID-19 Student Health and Expectations Course}. Before coming on to campus each day, all students are required to complete a \href{https://www.colorado.edu/protect-our-herd/daily-health-form}{Daily Health Form}. 

Students who have tested positive for COVID-19, have symptoms of COVID-19, or have had close contact with someone who has tested positive for or had symptoms of COVID-19 must stay home and complete the \href{https://www.colorado.edu/protect-our-herd/daily-health-form}{Health Questionnaire and Illness Reporting Form} remotely. In this class, if you are sick or quarantined,

\begin{itemize}
	\item If you are asymptomatic, please attend lecture via zoom.
	
	\item If you are symptomatic, please email me and we will plan how you can keep up with the material.
\end{itemize}



\newpage
\section*{Tentative Course Outline}

\footnotesize{
\begin{longtable}{|c|c|p{.78\textwidth}|}
	\hline
	\textbf{Week} & \textbf{Dates} & \textbf{Content} \\
	\hline
	1 & Aug 24-28 & \begin{minipage}{.85\textwidth}
		\begin{itemize} \itemsep-0.4em
			\vspace{1mm}
			\item Topics: Administration, Introduction to Statistics, Population vs. Sample
			\item Chapters: 1, 2 
			\item Due: \textcolor{green}{Homework 1, Ch. 1-2, due Sun at 11:59pm}
			\vspace{1mm}
		\end{itemize}
	\end{minipage} \\	
	\hline

	2 & Aug 31 - Sept 4 & \begin{minipage}{.85\textwidth}
		\begin{itemize} \itemsep-0.4em
			\vspace{1mm}
			\item Topics: Introduction to \textbf{\textsf{R}}, What is Probability, Random Variables, Probability Rules
			\item Chapters: 12, 13
			\item \textsf{R} Day: Monday 8/31 (Bring Laptop)
			\item Due: \textcolor{blue}{\textbf{\textsf{R} assignment 1, due Wed beginning of class of class}}
			\item Due: \textcolor{green}{Homework 2, Ch. 12-13, due Sun at 11:59pm}
			\item \textcolor{dark-maroon}{\textbf{No Class Labor Day, Monday 9/3}}
			\vspace{1mm}
		\end{itemize}
	\end{minipage} \\

	\hline
	3 & Sept 7-11 & \begin{minipage}{.85\textwidth}
		\begin{itemize} \itemsep-0.4em
			\vspace{1mm}
			\item Topics: Binomial Distribution
			\item Chapters: 14
			\item \textsf{R} Day: Wednesday 9/9 (Bring Laptop)
			\item Due: \textcolor{green}{Homework 3, Ch. 14, due Sun at 11:59pm}
			\vspace{1mm}
		\end{itemize}
	\end{minipage} \\
	
	\hline
	4 & Sept 14-18 & \begin{minipage}{.85\textwidth}
		\begin{itemize} \itemsep-0.4em
			\vspace{1mm}
			\item Topics: Normal Distribution, Distributions, and Mathematical Expectations
			\item Chapters: Chapter 3, Distributions Handout, Expectations Handout
			\item Due: \textcolor{blue}{\textbf{\textsf{R} assignment 2, due Wed beginning of class of class}} 
			\vspace{1mm}
		\end{itemize}
	\end{minipage} \\
	
	\hline
	5 & Sept 21-25 & \begin{minipage}{.85\textwidth}
		\begin{itemize} \itemsep-0.4em
			\vspace{1mm}
			\item Topics: Data Generation, \textbf{Midterm 1}
			\item Chapters: 8, 9
			\item Review Day: Friday 09/25
			\item Due: \textcolor{green}{Homework 4, Ch. 8-9, due Sun at 11:59pm}
			\vspace{1mm}
		\end{itemize}
	\end{minipage} \\
	
	\hline
	6 & Sept 28 - Oct 2 & \begin{minipage}{.85\textwidth}
		\begin{itemize} \itemsep-0.4em
			\vspace{1mm}
			\item Topics: Parameters and Statistics
			\item Chapters: 15
			\item \textcolor{red}{Midterm 1, Monday 9/28}
			\item Due: \textcolor{green}{Homework 5, Ch. 15, due Sun at 11:59pm}
			\vspace{1mm}
		\end{itemize}
	\end{minipage} \\
	
	\hline
	7 & Oct 5-9 & \begin{minipage}{.85\textwidth}
		\begin{itemize} \itemsep-0.4em
			\vspace{1mm}
			\item Topics: Confidence Intervals, Intro to Hypothesis Testing, $p$- values
			\item Chapters: 16, 17
			\item \textsf{R} Day: Monday 10/5 (Bring Laptop)
			\item Due: \textcolor{green}{Homework 6, Ch. 16-17, due Sun at 11:59pm}
			\vspace{1mm}
		\end{itemize}
	\end{minipage} \\
	
	\hline
	8 & Oct 12-16 & \begin{minipage}{\textwidth}
		\begin{itemize} \itemsep-0.4em
			\vspace{1mm}
			\item Topics: Size, Power, Inference
			\item Chapters: 18
			\item Due: \textcolor{blue}{\textbf{\textsf{R} assignment 3, due Wed beginning of class of class}}
			\item Due: \textcolor{green}{Homework 7, Ch. 18, due Sun at 11:59pm}
			\vspace{1mm}
		\end{itemize}
	\end{minipage} \\
	
	\hline
	9 & Oct 19-23 & \begin{minipage}{.85\textwidth}
		\begin{itemize} \itemsep-0.4em
			\vspace{1mm}
			\item Topics: $t$-distribution, Single and Two Sample Uses of $t$-distribution
			\item Chapters: 20, 21
			\item Due: \textcolor{green}{Homework 8, Ch. 20, due Sun at 11:59pm}
			\vspace{1mm}
		\end{itemize}
	\end{minipage} \\
	
	\hline
	10 & Oct 26-30 & \begin{minipage}{.85\textwidth}
		\begin{itemize} \itemsep-0.4em
			\vspace{1mm}
			\item Topics: Two Sample Uses of $t$-distribution, Review for Midterm 2
			\item Chapters: 21
			\item \textsf{R} Day: Monday 10/19 (Bring Laptop)
			\item Review Day: Wednesday 10/28
			\item Due: \textcolor{green}{Homework 9, Ch. 21, due Wednesday at 11:59pm}
			\item \textcolor{red}{\textbf{Midterm 2, Friday 10/30}}
			\vspace{1mm}
		\end{itemize}
	\end{minipage} \\

	
	\hline
	11 & Nov 2-6 & \begin{minipage}{.85\textwidth}
		\begin{itemize} \itemsep-0.4em
			\vspace{1mm}
			\item Topics: Tests of Proportions 
			\item Chapters: 22, 23
			\item Due: \textcolor{blue}{\textbf{\textsf{R} assignment 4, due Wed beginning of class of class}}
			\item Due: \textcolor{green}{Homework 10, Ch. 22-23, due Sun at 11:59pm}
			\vspace{1mm}
		\end{itemize}
	\end{minipage} \\

	\hline
	12 & Nov 9-13 & \begin{minipage}{.85\textwidth}
		\begin{itemize} \itemsep-0.4em
			\vspace{1mm}
			\item Topics: Covariance \& Correlation, Least Squares, Marginal and Conditional Probability
			\item Chapters: 4, 5, 6
			\item Due: \textcolor{green}{Homework 11, Ch. 4-5, due Sun at 11:59pm}
			\vspace{1mm}
		\end{itemize}
	\end{minipage} \\
	
	\hline
	13 & Nov 16-20 & \begin{minipage}{.85\textwidth}
		\begin{itemize} \itemsep-0.4em
			\vspace{1mm}
			\item Topics: Marginal and Conditional Probability
			\item Chapters: 6
			\item \textsf{R} Day: Wednesday 11/18 (Bring Laptop)
			\item No Class 11/20, Drop-in Meetings for \textbf{\textsf{R}} Project
			\item Going Virtual: 11/20
			\item Due: \textcolor{green}{Homework 12, Ch. 6, due Sun at 11:59pm}
			\vspace{1mm}
		\end{itemize}
	\end{minipage} \\
	
	\hline
	14 & Nov 23-27 & \begin{minipage}{.85\textwidth}
		\begin{itemize} \itemsep-0.4em
			\vspace{1mm}
			\item No Class 11/23, Drop-in Meetings for \textbf{\textsf{R}} Project
			\item \textcolor{dark-maroon}{Optional \textsf{R} Day: Wednesday 11/25 (ggplot2 for better graphing)}
			\item \textcolor{dark-maroon}{No Class Thanksgiving Break, Friday 11/27}
			\item Due: \textcolor{blue}{\textbf{\textsf{R} Project due Sunday at 11:59pm}}
			\vspace{1mm}
		\end{itemize}
	\end{minipage} \\
	
	\hline
	15 & Nov 30 - Dec 4 & \begin{minipage}{.85\textwidth}
		\begin{itemize} \itemsep-0.4em
			\vspace{1mm}
			\item Topics: Inference in Regression, Final Exam Review
			\item Chapters: 26
			\item Due: \textcolor{green}{Homework 13, Ch. 26, due Sun at 11:59pm}
			\vspace{1mm}
		\end{itemize}
	\end{minipage} \\
	
	\hline
	16 & Dec 7 & \begin{minipage}{.85\textwidth}
		\begin{itemize} \itemsep-0.4em
			\vspace{1mm}
			\item Topics: Final Exam Review
			\item Due: \textcolor{blue}{\textbf{\textsf{R} assignment 5, due Wed beginning of class of class}}
			\vspace{1mm}
		\end{itemize}
	\end{minipage} \\
	
	\hline
	& Dec 13 & \begin{minipage}{.85\textwidth}
		\begin{itemize} \itemsep-0.4em
			\vspace{5mm}
			\item \textcolor{red}{\textbf{Final Exam: Sunday, Dec 13 1:30pm -- 4:00pm}}
			\vspace{5mm}
		\end{itemize}
	\end{minipage} \\
	\hline
\end{longtable}
}


\newpage
\section*{University Policies}

\footnotesize{
	\vspace{5mm}\noindent 
	\textbf{Students with Disabilities:} 
	If you qualify for accommodations because of a disability, please submit to me a letter from disability services in a timely manner so that your needs can be addressed. Disability services determine accommodations based on documented disabilities.
	Contact: 303-492-8671, Center for Community N200.
	
	\vspace{5mm}\noindent
	\textbf{Religious Observance Policy:}
	Campus policy regarding religious observances requires that faculty make every effort to reasonably and fairly deal with all students who, because of religious obligations, have conflicts with scheduled exams, assignments, or required attendance. If you have a conflict, please contact me at the beginning of the term so we can make proper arrangements. 
	
	\vspace{5mm}\noindent
	\textbf{Honor Code:}
	All students of the University of Colorado at Boulder are responsible for knowing and adhering to the academic integrity policy of this institution. Violations of this policy may include: cheating, plagiarism, aid of academic dishonesty, fabrication, lying, bribery, and threatening behavior. All incidents of academic misconduct shall be reported to the Honor Code Council (\href{mailto:honor@colorado.edu}{honor@colorado.edu}; 303-725-2273).
	Students who are found to be in violation of the academic integrity policy will be subject to both academic sanctions from the faculty member and non-academic sanctions (including but not limited to university probation, suspension, or expulsion). Other information on the Honor Code can be found at: \url{http://www.colorado.edu/policies/honor.html} and at \url{http://www.colorado.edu/academics/honorcode/}
	
	\vspace{5mm}\noindent
	\textbf{Discrimination \& Harassment Policy:}
	The University of Colorado Policy on Sexual Harassment applies to all students, staff and faculty. Sexual harassment is unwelcome sexual attention. It can involve intimidation, threats, coercion, or promises or create an environment that is hostile or offensive. Harassment may occur between members of the same or opposite gender and between any combinations of members in the campus community: students, faculty, staff, and administrators. Harassment can occur anywhere on campus, including the classroom, the workplace, or a residence hall. Any student, staff or faculty member who believes s/he has been sexually harassed should contact the Office of Discrimination and Harassment (ODH) at 303-492-2127 or the Office of Judicial Affairs at 303-492-5550. Information about the ODH and the campus resources available to assist individuals who believe they have been sexually harassed can be obtained at: \url{http://www.colorado.edu/odh/}

}



\end{document}