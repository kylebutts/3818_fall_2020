\documentclass{beamer}
\usetheme{metropolis} % Use metropolis theme

\title{ECON 3818: Introduction to Statistics with Computer Applications}
%\subtitle
\date{\today}
\author{Kyle Butts}

\definecolor{blue}{RGB}{0,114,178}
\definecolor{red}{HTML}{EB0E09}
\definecolor{yellow}{RGB}{240,228,66}
\definecolor{green}{RGB}{0,158,115}
\definecolor{maroon}{HTML}{AF3335}
\definecolor{purple}{HTML}{7E90B8}

\definecolor{mybackground}{HTML}{ECECEC}
\setbeamercolor{background canvas}{bg= mybackground}

\definecolor{buff-gold}{HTML}{CFB87C}
\definecolor{buff-grey}{HTML}{565A5C}
\definecolor{buff-lightgrey}{HTML}{A2A4A3}
\definecolor{buff-black}{HTML}{000000}

\setbeamercolor{alerted text}{fg=buff-gold!80!black}
\setbeamercolor{frametitle}{bg=buff-black}
\setbeamercolor{title}{fg=buff-grey}
\setbeamercolor{button}{bg=buff-gold}

% Allow to remove indent w/ \begin{itemize}[leftmargin= *]
\usepackage{enumitem}
\setlist[itemize]{label= \textbullet}

% \usepackage[libertine]{newtxmath}
\usepackage{longtable}
\usepackage{booktabs}
\usepackage{enumitem}


\begin{document}

% Title Page ---------------------------------------
\maketitle


% Chapter 23 ---------------------------------------
\section{Chapter 23: Comparing Two Proportions}
\begin{frame}{Notation}
	We will use notation similar to that used in our study of two-sample t-statistics.
	
	\begin{center}
		\scalebox{0.9}{
			\begin{tabular}{|c|c|c|c|}
				\hline
				\textbf{Population} & \textbf{Pop. Proportion} & \textbf{Sample Size} & \textbf{Sample Proportion} \\ [0.5ex]
				\hline
				1 & $p_1$ & $n_1$ & $\hat{p}_1$ \\
				\hline
				2 & $p_2$ & $n_2$ & $\hat{p}_2$ \\
				\hline
			\end{tabular}
		}
	\end{center}
\end{frame}

\begin{frame}{Sampling Distribution of $\hat{p}$ Review}
	$X \sim B(1, p)$ is the underlying variable. 

	\[ \hat{p} = \frac{\sum \text{\# of successes}}{n} \]

	The sample distribution of $\hat{p}$ with population proportion $p_0$:

	\[ \hat{p} \sim N( p_0, \frac{p_0(1-p_0)}{n}) \]	
\end{frame}


\begin{frame}{Sampling Distribution of a Difference between Proportions}
	To use $\hat{p}_1 - \hat{p}_2$ for inference we use the following information:
	
	\begin{itemize}
		\item When the samples are large, the distribution of $\hat{p}_1 - \hat{p}_2$ is \alert{approximately normal}
		
		\item The \alert{mean} of the sampling distribution is: $p_1 - p_2$
		
		\item Assuming the two populations are independent, the \alert{standard deviation} of the distribution is: 
			\[ 
				\sqrt{\frac{p_1(1-p_1)}{n_1}+\frac{p_2(1-p_2)}{n_2}} 
			\]
	\end{itemize}
\end{frame}

\begin{frame}{Normal Distribution Review}
	If $X_1 \sim N(\mu_1, \sigma_1^2)$ and $X_2 \sim N(\mu_2, \sigma_2^2)$ are normally distributed and indepdent, then $X_1 - X_2$ is normally distributed,

	\[ E\{X_1 - X_2\} = \mu_1 - \mu_2, \]

	\[ Var(X_1 - X_2) = \sigma_1^2 + \sigma_2^2 \]
\end{frame}

\begin{frame}{Large-Sample Confidence Intervals for Comparing Proportions}
	Using the equation for standard error:
	\[ 
		SE = \sqrt{\frac{\hat{p}_1(1-\hat{p}_1)}{n_1} + \frac{\hat{p}_2(1-\hat{p}_2)}{n_2}} 
	\] 
	The confidence interval is constructed as:
	\[ 
		\hat{p}_1 - \hat{p}_2 \pm Z^* SE 
	\] 
\end{frame}

\begin{frame}{Example}
	Construct a 95\% confidence interval for the following difference in proportions:
	\begin{center}
		\begin{tabular}{|c|c|c|c|}
			\hline
			\textbf{Population} & \textbf{\# Successes} & \textbf{Sample Size} & \textbf{Sample Proportion} \\ [0.5ex]
			\hline
			1          & $75$         & $100$       & $\hat{p}_1=0.75$  \\
			\hline
			2          & $56$         & $100$       & $\hat{p}_2=0.56$  \\
			\hline
		\end{tabular}
	\end{center}
\end{frame}

\frame

\begin{frame}	
	Construct a 95\% confidence interval for the following difference in proportions:
	
	\begin{center}
		\begin{tabular}{|c|c|c|c|}
			\hline
			\textbf{Population} & \textbf{\# Successes} & \textbf{Sample Size} & \textbf{Sample Proportion} \\ [0.5ex]
			\hline
			1          & $75$         & $100$       & $\hat{p}_1=0.75$  \\
			\hline
			2          & $56$         & $100$       & $\hat{p}_2=0.56$  \\
			\hline
		\end{tabular}
	\end{center}

	\[ 
		SE=\sqrt{\frac{(0.75)(0.25)}{100}+\frac{(0.56)(0.44)}{100}}=0.0659
	\]

	Confidence interval = $(0.75-0.56)\pm(1.96)(0.0659) \implies 0.06 \text{ to } 0.32$
\end{frame}


\begin{frame}{Significance Tests for Comparing Proportions}
	\[ 
		H_0: p_1 - p_2 = 0 
	\]
	\[ 
		H_1: p_1 - p_2 \neq 0 
	\]
	In order to test the hypothesis, we must first calculated the \alert{pooled sample proportion}
	
	\[ \hat{p}=\frac{\text{number of successes in \textit{both samples combined}}}{\text{number of individuals in \textit{both samples combined}}} 
	\]
	Then we use the following z-statistic:
	
	\[ \frac{\hat{p}_1-\hat{p}_2}{\sqrt{\hat{p}(1-\hat{p})\big(\frac{1}{n_1}+\frac{1}{n_2}\big)}} 
	\]
\end{frame}

\begin{frame}{Example}
	\begin{center}
		\begin{tabular}{|c|c|c|c|}
			\hline
			\textbf{Population} & \textbf{\# Successes} & \textbf{Sample Size} & \textbf{Sample Proportion} \\ [0.5ex]
			\hline
			1 & $212$ & $616$ & $\hat{p}_1 = 0.344$ \\
			\hline
			2 & $7$ & $49$ & $\hat{p}_2 = 0.143$ \\
			\hline
		\end{tabular}
	\end{center}

	\pause \begin{itemize}
		\item Calculate \[ 
			\hat{p} = \frac{212+7}{616+49} = 0.329 
		\]
		\item Calculate Z-statistic
	\end{itemize}
	
	\[ 
		Z= \frac{0.344-0.143}{\sqrt{(0.329)(0.671) \left(\frac{1}{616}+\frac{1}{49}\right)} } =2.88
	\]
\end{frame}

\begin{frame}{Example -- continued}
	The z-statistic was 2.88, and we have a two-tailed alternative hypothesis. Therefore:
	\[ 
		\text{p-value }= 2\cdot P(Z>2.88) = 2\cdot 0.002 = 0.004 
	\]
	Therefore we reject null at $\alpha = 0.05$
\end{frame}






\end{document}