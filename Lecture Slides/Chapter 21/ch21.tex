\documentclass{beamer}
\usetheme{metropolis} % Use metropolis theme


\title{ECON 3818: Introduction to Statistics with Computer Applications}
%\subtitle
\date{\today}
\author{Kyle Butts}

\definecolor{blue}{RGB}{0,114,178}
\definecolor{red}{HTML}{EB0E09}
\definecolor{yellow}{RGB}{240,228,66}
\definecolor{green}{RGB}{0,158,115}
\definecolor{maroon}{HTML}{AF3335}
\definecolor{purple}{HTML}{7E90B8}

\definecolor{mybackground}{HTML}{ECECEC}
\setbeamercolor{background canvas}{bg= mybackground}

\definecolor{buff-gold}{HTML}{CFB87C}
\definecolor{buff-grey}{HTML}{565A5C}
\definecolor{buff-lightgrey}{HTML}{A2A4A3}
\definecolor{buff-black}{HTML}{000000}

\setbeamercolor{alerted text}{fg=buff-gold!80!black}
\setbeamercolor{frametitle}{bg=buff-black}
\setbeamercolor{title}{fg=buff-grey}
\setbeamercolor{button}{bg=buff-gold}

% Allow to remove indent w/ \begin{itemize}[leftmargin= *]
\usepackage{enumitem}
\setlist[itemize]{label= \textbullet}

% \usepackage[libertine]{newtxmath}
\usepackage{longtable}
\usepackage{booktabs}
\usepackage{enumitem}


\begin{document}

% Title Page ---------------------------------------
\maketitle


% Chapter 21 ---------------------------------------
\section{Chapter 21: Comparing Two Means}


\begin{frame}{Two-Sample Framework}
	Comparing two populations, or two treatments, is one of the most common situations in statistics. These are called \alert{two-sample problems}
	
	\begin{itemize}
		\item Can divide into groups, A and B
		      \begin{itemize}
		      	\item Examples: Women vs. Men, Econ Majors vs. Non-Econ Majors
		      \end{itemize}
		\item We want to know if they differ along some measurable margin
		      \begin{itemize}
		      	\item Example: salary, hours of homework per week
		      \end{itemize}
	\end{itemize}
	This is different from the matched pairs set up because:
	\begin{itemize}
		\item We have a separate sample for each group
		\item We cannot match the observations
	\end{itemize}
\end{frame}

\begin{frame}{Two-Sample Framework}
	Consider two groups, A and B. You have the following information for each group:
	
	\begin{center}
		\begin{tabular}{|c|c|c|c|}
			\hline
			\textbf{Population/Group} & \textbf{Sample Mean} & \textbf{Mean} & \textbf{Standard Deviation} \\
			\hline
			A & $\bar{X}_A$ & $\mu_A$ & $\sigma_A$ \\
			\hline
			B & $\bar{X}_B$ & $\mu_B$ & $\sigma_B$ \\
			\hline
		\end{tabular}
	\end{center}
		
	We use $\bar{X}_A$ and $\bar{X}_B$ to say something about $\mu_A - \mu_B$
	\begin{itemize}
		\item Construct a confidence interval for $\mu_A-\mu_B$
		\item Test the hypothesis $H_0: \mu_A - \mu_B = 0$
	\end{itemize}
\end{frame}

\begin{frame}{Conditions for Two-Sample Inference}
	We use $\bar{X}_A$ and $\bar{X}_B$ to say something about $\mu_A - \mu_B$
	
	\begin{itemize}
		\item We have two SRS's from two distinct populations
		\item The two samples are independent of one another
		\item We measure the same response variable for both samples
		\item Both populations are normally distributed
	\end{itemize}
	
	In practice, it is enough the distributions have similar shapes and that the data have no strong outliers.
\end{frame}

\begin{frame}{Distribution of $\bar{X}_A$ and $\bar{X}_B$}
	If $X_i \sim N\left(\mu_i, \frac{\sigma^2_i}{n}\right)$ for $i \in A, B$, then:
	
	\begin{itemize}
		\item $\bar{X}_A$ and $\bar{X}_B$ is normally distributed
		      
		\item $E[\bar{X}_A - \bar{X}_B] = E[\bar{X}_A] - E[\bar{X}_B]$
		      
		\item $V[\bar{X}_A - \bar{X}_B] = V[\bar{X}_A] + V[\bar{X}_B]$ (by independence)
	\end{itemize}
	
	This implies:
	\[ 
		\bar{X}_A - \bar{X}_B \sim N(\mu_A-\mu_B, \frac{\sigma^2_A}{n} + \frac{\sigma^2_B}{n})
	\]
\end{frame}

\begin{frame}{Distribution of $\bar{X}_A$ and $\bar{X}_B$}
	Therefore, \textbf{when $\sigma^2$ is known:}
	
	\[
		\frac{(\bar{X}_A-\bar{X}_B) - (\mu_A-\mu_B)}{\sqrt{\frac{\sigma^2_A}{n_A} + \frac{\sigma^2_B}{n_B}}} \sim N(0,1)
	\]
\end{frame}


\begin{frame}{Distribution of $\bar{X}_A$ and $\bar{X}_B$}
	As we mentioned, we don't always know the population variance, $\sigma^2$. 
	
	If we don't know these values, we can use the sample standard deviations $s_A$ and $s_B$ as estimators. 
	
	The standard error for the difference in sample means is:

	\[
		SE_{\bar{X}_A-\bar{X}_B}=\sqrt{\frac{s^2_A}{n_A}+\frac{s^2_B}{n_B}}
	\]
\end{frame}

\begin{frame}{Distribution of $\bar{X}_A$ and $\bar{X}_B$}
	This means that 
	\[
		\frac{\bar{X}_A-\bar{X}_B - (\mu_A-\mu_B)}{\sqrt{\frac{s^2_A}{n_A} + \frac{s^2_B}{n_B}}}
	\]
	
	can be approximated by the t-distribution, where the degrees of freedom is $min\{n_A, n_B\}-1$
	
	Statistical software can be more exact, but the formulas get complicated 
\end{frame}

\begin{frame}{Two-Sample Confidence Interval}{$\sigma^2$ Known}
	A level C confidence interval for $\mu_A-\mu_B$:
	\[ 
		(\bar{X}_A-\bar{X}_B) \pm Z_{\frac{1-C}{2}} \cdot \sqrt{\frac{\sigma^2_A}{n_A}+\frac{\sigma^2_B}{n_B}}
	\]
\end{frame}

\begin{frame}{Two-Sample Confidence Interval Example}{$\sigma^2 Known$}
	Say we have two groups -- athletes and non-athletes and we're asked to construct a 95\% confidence interval for the difference in GPA $\mu_A-\mu_{NA}$

	
	\begin{center}
		\scalebox{0.8}{
			\begin{tabular}{|c|c|c|c|}
				\hline
				\textbf{Group} & \textbf{Sample Mean}   & \textbf{Standard Deviation} & \textbf{Sample Size} \\ 
				\hline
				Athletes     & $\bar{X}=2.8$ & $\sigma=0.4$       & n=15        \\
				\hline
				Non-athletes & $\bar{X}=2.9$ & $\sigma=0.5$       & n=25        \\
				\hline
			\end{tabular}
		}
	\end{center}

	\[ 
		CI= (2.8-2.9) \pm Z_{0.025}\cdot \sqrt{\frac{0.4^2}{15}+\frac{0.5^2}{25}}
	\]
	\[ 
		\implies CI= [-0.38, 0.18]
	\]
\end{frame}

\begin{frame}{Two-Sample Confidence Interval}{$\sigma^2$ Unknown}
	\[ 
		(\bar{X}_A-\bar{X}_B) \pm t_{n-1}^{\frac{1-C}{2}} \cdot \sqrt{\frac{s^2_A}{n_A}+\frac{s^2_B}{n_B}}
	\]
\end{frame}

\begin{frame}{Two-Sample Confidence Interval Example}{$\sigma^2$ Unknown}
	We have 2 groups of students, and we're asked to construct 90\% confidence interval for difference in test scores, $\mu_A-\mu_B$
	\begin{center}
		\begin{tabular}{|c|c|c|c|}
			\hline
			\textbf{Group} & \textbf{Sample Mean}  & \textbf{Standard Deviation} & \textbf{Sample Size} \\
			\hline
			Group A & $\bar{X}$=76 & s=9 & n=60 \\
			\hline
			Group B  & $\bar{X}$=73 & s=5 & n=20 \\
			\hline
		\end{tabular}
	\end{center}

	\[
		CI=(76-73) \pm t^{0.05}_{19} \cdot \sqrt{\frac{9^2}{60}+\frac{5^2}{20}}
	\]
	\[
		CI=[0.21, 5.79]
	\]
\end{frame}

\begin{frame}{Two-Sample Hypothesis Testing: $\sigma^2$ Known}
	Researchers are asking college graduates how old they were when they had their first job. Researchers are curious to see if students who attended state schools got jobs earlier in life than those who attended private colleges.

	\begin{center}
		\begin{tabular}{|c|c|c|c|}
			\hline
			\textbf{Group} & \textbf{Sample Mean}  & \textbf{Standard Deviation} & \textbf{Sample Size} \\
			\hline
			A & $\bar{X}=18.19$ & $\sigma=3.8$ & $n = 20$ \\
			\hline
			\rule{0pt}{15pt} B & $\bar{X}=20.98$ & $\sigma=4.2$ & $n = 20$ \\
			\hline
		\end{tabular}
	\end{center}
	
	\[ 
		H_0: \mu_A - \mu_B = 0 
	\]
	\[ 
		H_1: \mu_A - \mu_B <0 
	\]
	
	Test the hypothesis at the $\alpha=0.05$ significance level
	
\end{frame}

\begin{frame}{Two-Sample Hypothesis Testing}{$\sigma^2$ Known}
	Calculate p-value using:
	\[
		P(\bar{X}_A-\bar{X}_B\leq 18.19-20.98 | \mu_A-\mu_B=0)
	\]
	
	\[ 
		P\left(\frac{\bar{X}_A-\bar{X}_B-(\mu_A-\mu_B)}{\sqrt{\frac{\sigma_A^2}{n_A} + \frac{\sigma_B^2}{n_B}}} \leq \frac{18.19-20.98 - (0)}{\sqrt{\frac{3.8^2}{20}+\frac{4.2^2}{20}}} \right)
	\]
	
	\vspace{5mm}
	$p$-value $= P(Z \leq -2.2) = 0.014 \implies \text{reject } H_0$ because p-value $\leq \alpha=0.05$
\end{frame}

\begin{frame}{Two-Sample Hypothesis Testing}{$\sigma^2$ Unknown}
	You want to test how attached individuals are to their friends, and whether that is different across people who volunteer for community service versus those who do not.

	\begin{center}
		\begin{tabular}{|c|c|c|c|}
			\hline
			\textbf{Group} & \textbf{Sample Mean}  & \textbf{Standard Deviation} & \textbf{Sample Size} \\
			\hline
			Service & $\bar{X} = 105.32$ & $s = 14.68$ & $n = 57$ \\
			\hline
			No service & $\bar{X} = 96.82$ & $s = 14.26$ & $n = 17$ \\
			\hline
		\end{tabular}
	\end{center}

	Test following hypothesis at $\alpha=0.01$ level:
	\[ 
		H_0: \mu_A - \mu_B = 0
	\]

	\[ 
		H_1: \mu_A - \mu_B \neq 0
	\]
\end{frame}

\begin{frame}{Two-Sample Hypothesis Testing}{$\sigma^2$ Unknown}
	\[ 
		t = \frac{\bar{X}_S-\bar{X}_{NS} - (\mu_A - \mu_B)}{\sqrt{\frac{s^2_S}{n_S} + \frac{s^2_{NS}}{n_{NS}}}} = \frac{105.32-96.82-(0)}{\sqrt{\frac{14.68^2}{57}+\frac{14.26^2}{17}}}
	\]

	\[ 
		t = \frac{8.5}{3.9677} = 2.142
	\]
	Look at t-table, row with degrees freedom = 16. 
	
	$t_{16}^{0.025} = 2.12$ and $t_{16}^{0.01} = 2.58$, this means p-value is in between 0.025 and 0.01, \textbf{BUT} it's a two-tailed test so we need to multiply these probabilities by 2:
	
	$0.02 < p$-value $< 0.05 \implies$ Do not reject null at $\alpha = 0.01$
\end{frame}

\begin{frame}{Review of Chapter 21}
	
	\begin{itemize}
		\item In this chapter, we focus on making inferences about the relationship between the means of two different samples
		      
			\begin{itemize}
		      	\item Confidence intervals around the difference in means $\mu_A - \mu_B \pm $ margin of error
		      	      
		      	\item Generally testing $H_0: \mu_A-\mu_B = 0$
			\end{itemize}
		      
		\item You'll be given sample means ($\bar{X}$), standard deviations ($\sigma$ or $s$) and population size ($n$) of each sample. 
			\begin{itemize}
		      	
		      	\item If you're given $\sigma$, use Z-distribution
		      	
		      	\item If you're given $s$, use t-distribution (unless \textbf{both} samples are large enough)
			\end{itemize}
	\end{itemize}
	
\end{frame}

\begin{frame}{Calculating Margin of Error with Two Samples}
	\[ 
		(\bar{X}_A-\bar{X}_B) \pm Z_{\frac{1-C}{2}} \cdot \sqrt{\frac{\sigma^2_A}{n_A}+\frac{\sigma^2_B}{n_B}}
	\]
	
	\[ 
		(\bar{X}_A-\bar{X}_B) \pm t_{n-1}^{\frac{1-C}{2}} \cdot \sqrt{\frac{s^2_A}{n_A}+\frac{s^2_B}{n_B}}
	\]
\end{frame}

\begin{frame}{Calculating Z-Statistic}
	\[ 
		Z = \frac{\bar{X}_A-\bar{X}_B-(\mu_A-\mu_B)}{\sqrt{\frac{\sigma_A^2}{n_A} + \frac{\sigma_B^2}{n_B}}} 
	\]
\end{frame}

\begin{frame}{Calculating t-Statistic}
	\[ 
		t = \frac{\bar{X}_S-\bar{X}_{NS}-(\mu_A-\mu_B)}{\sqrt{\frac{s^2_S}{n_S}+\frac{s^2_{NS}}{n_{NS}}}}
	\]
\end{frame}


\begin{frame}{Clicker Question}
	You're given the following information about average length of careers in NFL versus MLB.

	\begin{center}
		\begin{tabular}{|c|c|c|c|}
			\hline
			\textbf{Group} & \textbf{Sample Mean}  & \textbf{Standard Deviation} & \textbf{Sample Size} \\
			\hline
			NFL & $\bar{X} = 3.3$ & $\sigma = 2.1$ & $n = 20$ \\
			\hline
			MLB & $\bar{X} = 5.6$ & $\sigma=3.5$ & $n = 17$ \\
			\hline
		\end{tabular}
	\end{center}

	You want to construct a 90\% confidence interval. Given this information, calculate the margin of error:
	
	\begin{enumerate}[label=(\alph*)]
		\item 1.6
		
		\item 1.645
		
		\item 1.96
	\end{enumerate}
\end{frame}


\begin{frame}{Midterm Example}
	\footnotesize{New research has developed a new drug designed to reduce blood pressure. In an experiment, 21 subjects were assigned randomly to the treatment group and receive the experimental drug. The other 23 subjects were assigned to the control group and received a placebo treatment. A summary of these data is}

	\begin{center}
		\begin{tabular}{|c|c|c|c|}
			\hline 
			\textbf{Group} & \textbf{n} & $\bar{X}$ & \textbf{s} \\ 
			\hline
			Treatment group & 21 & 23.48 & 8.01 \\
			Control group   & 23 & 18.52 & 7.15 \\
			\hline
		\end{tabular}
	\end{center}

	We want to test whether there was any difference in means across these two groups:
	
	\begin{enumerate}[label=(\alph*)]
		\item State the null and alternative hypothesis 
		\item Calculate p-value or range of p-values
		\item Do you reject at $\alpha=0.05$ level?
		\item If you were incorrect in part c, what kind of error did you make? 
	\end{enumerate}
\end{frame}






\end{document}