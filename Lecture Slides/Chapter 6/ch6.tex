\documentclass{beamer}
\usetheme{metropolis} % Use metropolis theme

\title{ECON 3818: Introduction to Statistics with Computer Applications}
\date{\today}
\author{Kyle Butts}

\definecolor{blue}{RGB}{0,114,178}
\definecolor{red}{HTML}{EB0E09}
\definecolor{yellow}{RGB}{240,228,66}
\definecolor{green}{RGB}{0,158,115}
\definecolor{maroon}{HTML}{AF3335}
\definecolor{purple}{HTML}{7E90B8}
\definecolor{buff-gold}{HTML}{CFB87C}
\definecolor{buff-grey}{HTML}{565A5C}
\definecolor{buff-lightgrey}{HTML}{A2A4A3}
\definecolor{buff-black}{HTML}{000000}

\definecolor{mybackground}{HTML}{ECECEC}
\setbeamercolor{background canvas}{bg= mybackground}

\setbeamercolor{alerted text}{fg=buff-gold!80!black}
\setbeamercolor{frametitle}{bg=buff-black}
\setbeamercolor{title}{fg=buff-grey}
\setbeamercolor{button}{bg=buff-gold}

% Allow to remove indent w/ \begin{itemize}[leftmargin= *]
\usepackage{enumitem}
\setlist[itemize]{label= \textbullet}

% \usepackage[libertine]{newtxmath}
\usepackage{longtable}
\usepackage{booktabs}
\usepackage{enumitem}

\usepackage{diagbox}


\begin{document}

% Title Page ---------------------------------------
\maketitle




% Chapter 6 ----------------------------------------
\section{Chapter 6: Two-Way Tables}
\begin{frame}{Introduction}
	This chapter discusses the relationship between two categorical variables
	\begin{itemize}
		\item To analyze categorical data, we use the \textit{counts} or \textit{percentages} of individuals that fall into various categories
	\end{itemize}
\end{frame}

\begin{frame}{Two-Way Table}
	Typically, published data is grouped in order to save space. This chapter will talk about \alert{Two-Way Tables}
	
	
	\begin{center}
		\footnotesize{
			\begin{tabular}{|c|c|c|c|c|}
				\hline
				\diagbox{Sex}{Degree} & Associate & Bachelor's & Master's & Doctorate \\
				\hline
				Women                           & 673       & 1050       & 481      & 95        \\
				\hline
				Men                             & 401       & 780        & 342      & 89        \\
				\hline
			\end{tabular}}
	\end{center}
	This two-way table describes two categorical variables. One is the sex of an individual, this is the \alert{row variable} and the other is the degree attained, which is the \alert{column variable}
\end{frame}

\begin{frame}{Joint Distribution}
	The joint distribution is found by dividing each cell by the total count.
	$$P(x,y)=\frac{\text{number of times x and y occurs}}{\text{total number of occurrences}}$$
	Since we have 3911 people in total, we get the following joint probabilities:
	\begin{center}
		\footnotesize{
			\begin{tabular}{|c|c|c|c|c|}
				\hline
				\diagbox{Sex}{Degree} & Associate      & Bachelor's    & Master's      & Doctorate      \\
				\hline
				Women                           & 673            & 1050          & 481           & 95             \\
				Joint \%                        & \textbf{0.17}  & \textbf{0.27} & \textbf{0.12} & \textbf{0.024} \\ 
				\hline
				Men                             & 401            & 780           & 342           & 89             \\
				Joint \%                        & \textbf{0.103} & \textbf{0.2}  & \textbf{0.09} & \textbf{0.023} \\
				\hline
			\end{tabular}}
	\end{center}
	This means the probability of being a woman \textbf{and} having a Master's degree is 12\%
\end{frame}

\begin{frame}{Marginal Distribution}
	A \alert{marginal distribution} is the probability distribution associated with only one of the random variables
	
	In order to calculate, we need to look at the distribution of each variable \textit{separately}. We do this by looking at the "Total" column and "Total" row.
	 
	We will have two different marginal distributions, the "row" marginal and the "column" marginal
\end{frame}

\begin{frame}{Row Marginal}
	In this scenario, the row marginal is the distribution of sex alone:
	\begin{center}
		\scriptsize{
			\begin{tabular}{|c|c|c|c|c|c|}
				\hline
				\diagbox{Sex}{Degree} & Associate's    & Bachelor's    & Master's      & Doctorate      & Row Marginal   \\
				\hline
				Women                           & 673            & 1050          & 481           & 95             &                \\
				Joint \%                        & \textbf{0.17}  & \textbf{0.27} & \textbf{0.12} & \textbf{0.024} & \textbf{0.584} \\ 
				\hline
				Men                             & 401            & 780           & 342           & 89             &                \\
				Joint \%                        & \textbf{0.103} & \textbf{0.2}  & \textbf{0.09} & \textbf{0.023} & \textbf{0.416} \\
				\hline
			\end{tabular}}
	\end{center}
	58.4\% of individuals in this sample of degree holders are women.
\end{frame}

\begin{frame}{Column Marginal}
	In this scenario, the column marginal is the distribution of degrees alone:
	\begin{center}
		\scriptsize{
			\begin{tabular}{|c|c|c|c|c|}
				\hline
				\diagbox{Sex}{Degree} & Associate's    & Bachelor's    & Master's      & Doctorate       \\
				\hline
				Women                           & 673            & 1050          & 481           & 95              \\
				Joint \%                        & \textbf{0.17}  & \textbf{0.27} & \textbf{0.12} & \textbf{0.024}  \\ 
				\hline
				Men                             & 401            & 780           & 342           & 89              \\
				Joint \%                        & \textbf{0.103} & \textbf{0.2}  & \textbf{0.09} & \textbf{0.023}  \\
				\hline
				Column marginal                 & \textbf{0.273} & \textbf{0.47} & \textbf{0.21} & \textbf{ 0.047} \\
				\hline
			\end{tabular}}
	\end{center}
	The probability of an individual having a Bachelor's degree is 47\%.
\end{frame}

\begin{frame}{Conditional Distribution}
	
	We can use these tables to back out conditional distributions. 
	
	Remember:
	$$P(A \vert B)=\frac{P(A \cap B)}{P(B)}$$
	This is the same expression as:
	$$P(x\vert y)=\frac{P(x,y)}{P(y)}$$
	
\end{frame}

\begin{frame}{Conditional Distribution}
	For example, we can calculate the probability of holding each degree, given the individual is a woman: 
	
	P(Associate's $\vert$ Woman) = \\
	P(Associate's and Woman)/P(Woman)=$0.17/0.584=29.1\%$
	
	P(Bachelor's $\vert$ Woman) = \\
	P(Bachelor's and Woman)/P(Woman)=$0.27/0.584=46.2\%$
	
	P(Master's $\vert$ Woman) = \\
	P(Master's and Woman)/P(Woman)=$0.12/0.584=20.5\%$
	
	P(Doctorate $\vert$ Woman) = \\
	P(Doctorate and Woman)/P(Woman)=$0.024/0.584=4.1\%$
\end{frame}
%\begin{center}
%\begin{tabular}{|M{2cm}|M{7cm}|}
%\hline 
%Associate's &  $\frac{673}{2299} \rightarrow P(Associate's\vert woman)=29.3\%$ \\ [5pt]
%\hline
%Bachelor's &   $\frac{1050}{2299} \rightarrow P(Bachelor's\vert woman)=45.7\%$ \\ [5pt]
%\hline
%Master's &   $\frac{481}{2299} \rightarrow P(Master's\vert woman)=20.9\%$ \\ [5pt]
%\hline
%Doctorate &   $\frac{95}{2299} \rightarrow P(Doctorate\vert woman)=4.1\%$ \\ [5pt]
%\hline
%\end{tabular}
%\end{center}
%
%\end{frame}

\begin{frame}{Conditional Distribution}
	We could also calculate the probability an individual is a particular sex, based of holding an Associate's degree
	
	P(Male $\vert$ Associate's) = \\
	P(Male and Associate's)/P(Associate's)=$.103/0.273=37.7\%$
	
	P(Female $\vert$ Associate's) = \\
	P(Female and Associate's)/P(Associate's)=$.17/.273=62.3\%$
\end{frame}


\begin{frame}{Joint Probabilities vs. Conditional Probabilities}
	\begin{itemize}
		\item Joint probabilities take into account the probability that each event happens on its own
		\item Conditional probabilities assume that one event has already happened
	\end{itemize}
\end{frame}

\begin{frame}{Joint Probability vs. Conditional Probability}{Example}
	\begin{itemize}
		\item P(work in tech job $\cap$ live in Boulder) vs. \\ P(work in tech job $\vert$ live in Boulder)
		      \begin{itemize}
		      	\item P(work in tech) = work in tech/entire US population = relatively small, let's say 7\%
		      	\item P(live in Boulder) = Boulder population/entire US population = also small, $<$1\%
		      \end{itemize}
		\item This means the probability of BOTH happening is small, because both events are unlikely compared to the state space of the entire US population
		\item But the P(work in tech $\vert$ live in Boulder) will be higher because now the state space is Boulder population, which has a greater concentration of high-tech employees
	\end{itemize}
\end{frame}

\begin{frame}{Marginal and Conditional Distributions}
	Here are some formal definitions:
	The \alert{\small{marginal distribution}} of one of the categorical variables in a two-way table of counts is the \alert{\small{distribution of that variable among all individuals described by the table}}
	\begin{itemize}
		\item distribution of sex or degrees alone
	\end{itemize}
	
	A \alert{\small{conditional distribution}} of a variable is the \alert{\small{distribution of values of that variable among only individuals who have a given value of the other variable}}. There is a separate conditional distribution for each value of the other variable. 
	\begin{itemize}
		\item there are two sets of conditional distributions for any two-way table, probability of having a degree based on sex, probability of being a particular sex based on degree held
	\end{itemize}
\end{frame}

\begin{frame}{Clicker Question} 
	\begin{center}
		\begin{tabular}{|c|l|c|c|c|}
			\hline 
			& \multicolumn{4}{c|}{{\color{blue}Class of Person on Titanic}}   \\ 
			\hline 
			{\color{blue}Survival} & First & Second & Third & Crew \\ 
			\hline 
			Alive                  & 203   & 118    & 178   & 212  \\ 
			\hline 
			Dead                   & 122   & 167    & 528   & 673  \\ 
			\hline 
			%		{\color{blue}Total} & \textbf{325} & \textbf{285} & \textbf{706} &\textbf{ 885} & \textbf{2201} \\ 
			%		\hline 
		\end{tabular} 
	\end{center}
	
	Given this joint distribution, what is the probability of survival, given you are a first class passenger?
	\begin{enumerate}[label=(\alph*)]
		\item 9\%
		\item 62\%
		      \item1.3\%
		\item 30\%
	\end{enumerate}
\end{frame}

\begin{frame}{Simpson's Paradox}
	
	\url{https://ed.ted.com/lessons/how-statistics-can-be-misleading-mark-liddell}
	
\end{frame}

\begin{frame}{Simpson's Paradox}
	Consider the survival rates for the following groups of victims who were taken to the hospital, either by helicopter, or by road:
	
	{\footnotesize
	\begin{tabular}{|l|c|c|}
		\hline
		\textbf{Counts} & \textbf{Helicopter} & \textbf{Road} \\
		\hline \hline 
		Victim Died & 64 & 260 \\ \hline 
		Victim Survived & 136 & 840 \\ \hline 
		Total & 200 & 1100 \\ \hline 
	\end{tabular} 
	}
	{\footnotesize
	\begin{tabular}{|l|c|c|}
		\hline 
		\textbf{Percents} & \textbf{Died} & \textbf{Survived} \\ 
		\hline \hline
		Helicopter & 32\% & 68\% \\ \hline
		Road & 24\% & 76\% \\ \hline
	\end{tabular}
	}
	
	\vspace{5mm}
	The probability of died conditional on helicopter, is higher than the probability of died conditional road. Does this mean that this (more costly) mode of transportation isn't helping?
\end{frame}

\begin{frame}{Simpson's Paradox}
	The idea is there a confounding variable, the severity of the accident, behind the Statistics

	\begin{center}
		\textbf{Serious Accidents}
	\end{center}
	{\footnotesize
	\begin{tabular}{|l|c|c|}
		\hline
		\textbf{Counts} & \textbf{Helicopter} & \textbf{Road} \\
		\hline \hline 
		Victim Died & 48 & 60 \\ \hline 
		Victim Survived & 52 & 40 \\ \hline 
		Total & 100 & 100 \\ \hline 
	\end{tabular} 
	}
	{\footnotesize
	\begin{tabular}{|l|c|c|}
		\hline 
		\textbf{Percents} & \textbf{Died} & \textbf{Survived} \\ 
		\hline \hline
		Helicopter & 48\% & 52\% \\ \hline
		Road & 60\% & 40\% \\ \hline
	\end{tabular}
	}

	\begin{center}
		\textbf{Less Serious Accidents}
	\end{center}
	{\footnotesize
	\begin{tabular}{|l|c|c|}
		\hline
		\textbf{Counts} & \textbf{Helicopter} & \textbf{Road} \\
		\hline \hline 
		Victim Died & 16 & 200 \\ \hline 
		Victim Survived & 84 & 800 \\ \hline 
		Total & 200 & 1100 \\ \hline 
	\end{tabular} 
	}
	{\footnotesize
	\begin{tabular}{|l|c|c|}
		\hline 
		\textbf{Percents} & \textbf{Died} & \textbf{Survived} \\ 
		\hline \hline
		Helicopter & 16\% & 84\% \\ \hline
		Road & 20\% & 80\% \\ \hline
	\end{tabular}
	}

\end{frame}

\begin{frame}{Clicker Question}
	From 2010-2013 the US median wage increased 1\%, however over the same time period the median wage has decreased within each education subgroup (high school drop outs, high school graduates, some college, bachelor's or more).
	
	Which of the following explanations is consistent with Simpson's paradox?
	\begin{enumerate}[label=(\alph*)]
		\item The BLS didn't control for inflation
		\item There are more people with bachelor's degrees (high income people)
		\item The wage of the highest income earner went up even more
		\item There are less unemployed people now
	\end{enumerate}
\end{frame}



\end{document}