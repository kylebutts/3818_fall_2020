\documentclass{beamer}
\usetheme{metropolis} % Use metropolis theme

\title{ECON 3818: Introduction to Statistics with Computer Applications}
%\subtitle
\date{\today}
\author{Kyle Butts}

\definecolor{blue}{RGB}{0,114,178}
\definecolor{red}{HTML}{EB0E09}
\definecolor{yellow}{RGB}{240,228,66}
\definecolor{green}{RGB}{0,158,115}
\definecolor{maroon}{HTML}{AF3335}
\definecolor{purple}{HTML}{7E90B8}

\definecolor{mybackground}{HTML}{ECECEC}
\setbeamercolor{background canvas}{bg= mybackground}

\definecolor{buff-gold}{HTML}{CFB87C}
\definecolor{buff-grey}{HTML}{565A5C}
\definecolor{buff-lightgrey}{HTML}{A2A4A3}
\definecolor{buff-black}{HTML}{000000}

\setbeamercolor{alerted text}{fg=buff-gold!80!black}
\setbeamercolor{frametitle}{bg=buff-black}
\setbeamercolor{title}{fg=buff-grey}
\setbeamercolor{button}{bg=buff-gold}

% Allow to remove indent w/ \begin{itemize}[leftmargin= *]
\usepackage{enumitem}
\setlist[itemize]{label= \textbullet}

% \usepackage[libertine]{newtxmath}
\usepackage{longtable}
\usepackage{booktabs}
\usepackage{enumitem}


\begin{document}

% Title Page ---------------------------------------
\maketitle




% Chapter 14 ---------------------------------------

\section{Chapter 14: Binomial Distribution}
\begin{frame}{Binomial Distribution}
	
	\begin{itemize}
		\item Probability model used when there are two outcomes: success or failure
		      \begin{itemize}
		      	\item One trial -- \alert{Bernoulli distribution}
		      	\item Multiple trials -- \alert{Binomial distribution}
		      \end{itemize}
		\item Examples:
		      \begin{itemize}
		      	\item Flipping a coin, shooting a free throw
		      \end{itemize}
	\end{itemize}
	
\end{frame}

\begin{frame}{Bernoulli Distribution}
	
	\begin{itemize}
		\item If X follows a Bernoulli process, then we say that $$X \sim B(1,p)$$
		      \begin{itemize} 
		      	\item where 1 tells us the number of trials
		      	\item and p tells us the probability of success
		      	\item $\sim$: "is distributed"
		      \end{itemize}
	\end{itemize}
	
\end{frame}

\begin{frame}{Binomial Distribution}
	
	Generally, we focus on situations where we have more than one trial so we use the \alert{binomial distribution}
	\begin{itemize}
		\item \underline{The Binomial Setting}
		      \begin{itemize}
		      	\item There are a fixed number, n, trials 
		      	\item These n trials are all independent
		      	\item Each trial falls into one of just two categories -- "success" or "failure"
		      	\item The probability of success, p, is the same for each observations
		      \end{itemize}
		\item Notation, $X \sim B(n,p)$
		      \begin{itemize}
		      	\item where n, p are \textit{parameters}
		      \end{itemize}
	\end{itemize}
	
\end{frame}

\begin{frame}{Binomial Example}
	
	Shaquille O'Neal is allowed 5 free throw shots. He has a 60\% chance of making each shot. What is the probability he makes 3 out of the 5 shots?
	\begin{itemize}
		\item Setting up a binomial:
		      \begin{itemize} 
		      	\item There are fixed number of trials,\checkmark yes, n=5
		      	\item These trials are independent \checkmark 
		      	\item Each trial falls into one of two categories \checkmark yes, miss or make
		      	\item Probability it the same for each trial \checkmark yes, p=0.6
		      \end{itemize}
	\end{itemize}
	
\end{frame}

\begin{frame}{Clicker Question}
	
	For which of the following counts would a binomial probability model be reasonable?
	
	\begin{enumerate}[label=(\alph*)]
		\item the number of phone calls received in a one-hour period
		\item the number of hearts you draw when you select 5 cards from a standard deck of 52 cards
		\item the number of sevens in a randomly selected set of five digits from a table of random digits
	\end{enumerate}
	
\end{frame}



\begin{frame}{Binomial Example}
	
	\textbf{Important: There is more than one way to make 3 out 5 free throws}
	\begin{columns}
		\column{0.5\textwidth}
		\footnotesize{
			\begin{enumerate}
				\item[1.] Make, Make, Make, Miss, Miss
				\item[2.] Make, Make, Miss, Miss, Make
				\item[3.] Make Make, Miss, Make, Miss
				\item[4.] Make, Miss, Miss, Make, Make
				\item[5.] Make, Miss, Make, Miss, Make
			\end{enumerate}}
		\column{0.5\textwidth}
		\footnotesize{
			\begin{enumerate}
				\item[6.] Miss, Make, Make, Make, Miss
				\item[7.] Miss, Make, Make, Miss, Make
				\item[8.] Miss, Make, Miss, Make, Make
				\item[9.] Miss, Make, Make, Make, Miss
				\item[10.] Miss, Miss, Make, Make, Make
			\end{enumerate}}
	\end{columns}
	
\end{frame}

\begin{frame}{Binomial Coefficient}
	
	First step to solving a binomial probability is to calculate the number of different ways of getting exactly that many success in n observations.
	\begin{itemize} 
		\item The number of ways of getting k successes in n trials is given by the \alert{Binomial Coefficient}: \[ 
			{n \choose k}= \frac{n!}{k!(n-k)!}
		\]
	
		\item ${n \choose k}$ is read "n choose k" which means "how many different ways to get k successes in n trials" (you can google n choose k and the calculator will tell you)
		
		\item n! = $n*(n-1)*(n-2)...(3)*(2)*(1)$
		      \begin{itemize}
		      	\item Example: 5! = $5*4*3*2*1$=120
		      	\item Note: 0! = 1
		      \end{itemize}
	\end{itemize}
	
\end{frame}

\begin{frame}{Binomial Formula}
	
	Formula tells us the probability of getting k successes in n trials:
	$$P(X=k) = {n \choose k} p^k (1-p)^{n-k}$$
	\begin{itemize}
		\item ${n \choose k}$: number of ways to get k successes in n trials
		\item $p^k$: probability of success, raised by the number of successes
		\item $(1-p)^{n-k}$: probability of failure, raised by the number of failures
	\end{itemize}
	
\end{frame}

\begin{frame}{Binomial Example}
	
	Back to our previous example. Shaq is shooting 5 free throws and has a 60\% chance of making each one. What is the probability he makes 3?
	$$P(X=3) = {5 \choose 3}*0.6^3*0.4^2$$
	$$P(X=3) = 0.3456$$
	
\end{frame}

\begin{frame}{Clicker Question}
	
	What is the probability of making 4 out of 6 penalty kicks if the probability of scoring is 70\%?
	
	\begin{enumerate}[label=(\alph*)]
		\item 70\%
		\item 2.2\%
		\item 32.4\%
		\item 65.47\%
		      
	\end{enumerate}
\end{frame}

\begin{frame}{Binomial Probabilities}
	
	\begin{itemize}
		\item The binomial formula
		$$P(X=k) = {n \choose k} p^k (1-p)^{n-k}$$
		only calculates the probability that X is equal to one specific (discrete) number.
		
		\item In order to calculate the probability that X takes on multiple value means we must use this formula repeatedly
		 
		\item What if we wanted to know the probability that Shaq makes at least two of the five free throws?
		
	\end{itemize}
	
\end{frame}

%\begin{frame}{Example}
%
%A local veterinary clinic typically sees 15\% of its horses presenting with West Nile virus. If 10 horses are admitted during July, what is the probability that 2 or fewer horses among the 10 horses admitted have been infected with West Nile virus?

%
%\end{frame}

\begin{frame}{Cumulative Binomial Probabilities}
	
	Recall the probability rule:
	$$P(A^c)=1-P(A)$$
	We can use this when calculating binomial probabilities. Make sure to pay attention to the wording!
	
	\begin{itemize}
		\item \alert{At least} is inclusive. 
		      \begin{itemize}
		      	\item P(at least 1) = $P(X\geq1)=1-P(X=0)$
		      \end{itemize}
		\item \alert{More than} is not inclusive
		      \begin{itemize}
		      	\item P(more than 1)=$P(X>1)=1-P(X=0)-P(X=1)$
		      \end{itemize}
	\end{itemize}
	
\end{frame}

\begin{frame}{Cumulative Example}
	What is the probability that Shaq makes at least two out of his five free throws?
	$$P(X\geq2)=1-P(X=0)-P(X=1)$$
	\begin{itemize}
		\item $P(X=0)= {5 \choose 0}*0.6^0*0.4^5 = 0.010$
		\item $P(X=1)= {5 \choose 1}*0.6^1*0.4^4 = 0.077$
		\item $1 - 0.01 - 0.077 = 0.913$
	\end{itemize}
\end{frame}

\begin{frame}{Clicker Question}
	
	You are asking someone out on a date. Your probability of success is 35\% each time you try. If you ask out 4 people, what are the odds that you get at least one yes?
	
	\begin{enumerate}[label=(\alph*)]
		\item 98.5\%
		\item 1.5\%
		\item 82.15\%
		\item 17.85\%
	\end{enumerate}
	
\end{frame}


\begin{frame}{Binomial Mean and Standard Deviation}
	
	If $X \sim B(n,p)$, the \alert{mean} and \alert{standard deviation} of X are:
	\begin{itemize}
		\item $\mu = np$
		\item $\sigma = \sqrt{np(1-p)}$
	\end{itemize}
	
\end{frame}

\begin{frame}{Clicker Question (Midterm Example)}
	If X has a binomial distribution with 20 trials and a mean of 6, then the success probability, $p$, is:
	\begin{enumerate}[label=(\alph*)]
		\item 0.3
		\item 0.5
		\item 0.75
		\item Cannot be determined given the information
	\end{enumerate}
\end{frame}

\begin{frame}{Recognizing the Binomial Setting}
	Which of the three scenarios would it be reasonable to use a binomial distribution for random variable X?
	\begin{itemize}
		\item An auto manufacturer chooses one car from each hour's production for a quality inspection. The variable X is the count of defects in the car's paint
		      % Here X could be more than 2 values -- 1,2,3, etc. Any number of defects. An example where X would be binomial would be X =1 if there are any defects; 0 if no defects
		\item The pool of potential jurors for a murder case contains 100 people chosen at random. Each person in in the pool is asked whether they oppose the death penalty. X is the number of people who say yes
		      % Yes, here X can only be two values, 1 if approve death penalty 0 if disapprove. Then p would be the probability an individual picked at random approves of the death penalty
		\item Joe buys a ticket in his state's lottery game every week. X is the number of times in that year that he receives a prize.
		      % Yes, here X can be two values. 1 if Joe wins anything 0 if Joe wins nothing. So p would be the porbability that Joe wins ANY prize on lottery ticke.t
	\end{itemize}
\end{frame}

\begin{frame}{Example}
	
	You're taking an exam consisting of 8 multiple choice questions, each with 4 available answers. You forgot to study, so you have to guess on every question. What is the probability you only get 2 questions wrong?
	
	\begin{enumerate}[label=(\alph*)]
		\item 0.3114
		\item 0.0038
	\end{enumerate}
	
\end{frame}


%\begin{frame}{Example}
%A very large gardening business grows rose bushes for sale to garden stores around the world. The most popular %colors are red, pink, and white. The business decides on 50\% red roses, 30\% pink, and 20\% white. A gardener %orders 9 rose bushes selected randomly from a huge field. Her primary interest is in pink roses. Assuming rose %bushes are selected independently, what is the probability of getting at least 3 pink rose bushes?
% 
%$$P(X\geq3)=0.534$$
%\end{frame}





\end{document}